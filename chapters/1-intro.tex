\part{简介与准备工作}
\chapter{课程简介}
\section{关于 STM32}
STM32 是指意法半导体 (STMicroelectronics) 公司开发的 32 位微控制器 (Microcontroller), 广泛应用于嵌入式系统和物联网设备. 该系列微控制器基于 ARM Cortex-M 核心, 提供从 Cortex-M0 到 Cortex-M7 等多个性能等级, 满足不同应用需求.

\subsection{性能与架构}

STM32 微控制器家族的主要特点在于其高性能与低功耗的平衡, 该系列微控制器采用多种 ARM Cortex-M 核心, 涵盖了从低端到高端的各种应用需求. Cortex-M0 和 M0+ 内核主要用于低功耗和低成本的应用, 如传感器, 简单控制系统和可穿戴设备. 这些内核的设计目标是提供足够的计算能力, 同时最大限度地延长电池寿命. Cortex-M3 和 M4 内核在性能和功耗之间取得了更好的平衡, 适用于中等复杂度的任务, 如家电控制, 工业自动化和医疗设备. 这些内核不仅具备较高的计算能力, 还支持丰富的外设接口. Cortex-M7 内核提供了最高的性能, 具有高达几百 MHz 的主频, 适用于需要复杂计算和实时处理的应用, 如多媒体设备, 图像处理和高级控制系统. 其强大的处理能力和浮点运算单元使其在需要高性能计算的场合表现出色.

\subsection{外设与集成度}

STM32 系列微控制器集成了丰富的外设, 使其能够满足各种嵌入式系统的需求. 每个 STM32 微控制器通常都配备了多个 ADC (模数转换器), 支持高精度的模拟信号采集. DAC (数模转换器) 则用于生成精确的模拟输出信号. 定时器和 PWM (脉宽调制) 模块提供了灵活的时间和信号控制功能, 适用于电机控制和信号调制. 通信接口如 USART, I2C, SPI 等支持各种串行通信标准, 确保与外部设备的高效数据交换. 丰富的 GPIO (通用输入输出) 引脚允许用户根据需要配置输入或输出功能. 这些外设的高度集成使得 STM32 微控制器在减少外部元件, 降低系统成本和提高系统可靠性方面具有明显优势.

\subsection{低功耗特性}

低功耗设计是 STM32 的重要特性之一, 这使其在需要电池供电或能量采集的应用中表现尤为突出. STM32 微控制器支持多种低功耗模式, 例如睡眠模式, 停止模式和待机模式, 可以在不同的应用场景下实现最优的能耗管理. 睡眠模式下, 处理器暂停运行, 但保持外设的工作状态, 从而快速响应中断事件. 停止模式则进一步降低功耗, 通过停止所有时钟和大部分外设, 只保留关键部分的运行. 待机模式是最低功耗模式, 仅保留少量的内存和外设状态, 以确保快速恢复. STM32L 系列微控制器特别针对低功耗应用进行了优化, 例如通过使用超低功耗的制程工艺和专有的节能技术, 使其非常适用于便携设备, 能量采集系统和长寿命传感器节点.

\subsection{开发生态系统}

STM32 的成功还离不开其强大的开发生态系统. 意法半导体为开发者提供了一整套开发工具和软件资源, 显著简化了开发过程. STM32CubeMX 是一款图形化配置工具, 允许用户直观地配置外设, 时钟和引脚映射. STM32Cube HAL (硬件抽象层) 库提供了一组简化的 API, 帮助开发者更方便地访问硬件功能. STM32CubeIDE 是一个集成开发环境, 结合了编译, 调试和项目管理功能, 为开发者提供了一站式开发平台. 此外, STM32 还支持多种第三方开发工具, 如 Keil, IAR 和 GNU 工具链, 进一步增加了开发的灵活性. 支持的实时操作系统 (RTOS) 包括 FreeRTOS, mbed OS 和 Zephyr OS, 使得 STM32 在实时应用中的表现更加出色. 这些工具和资源形成了一个完整的生态系统, 极大地提高了开发效率和产品上市速度.

\subsection{应用领域}

得益于其性能, 功能和低功耗特性, STM32 微控制器被广泛应用于多个领域. 家用电器领域, STM32 用于控制智能家居设备, 如智能灯泡, 恒温器和家电控制面板, 提供稳定和高效的控制功能. 工业自动化中, STM32 微控制器被用于 PLC (可编程逻辑控制器), 工业机器人和传感器接口, 提升了系统的自动化水平和可靠性. 在医疗设备中, STM32 用于便携式医疗仪器, 监护设备和健康监测设备, 保证了精确的数据采集和处理. 消费电子方面, STM32 被用于智能手表, 健身追踪器和多媒体播放器, 提供高性能和低功耗的解决方案. 汽车电子领域, STM32 用于车载娱乐系统, 传感器接口和电动汽车控制单元, 满足了汽车行业对可靠性和性能的高要求. 物联网 (IoT) 设备中, STM32 的低功耗和丰富外设使其成为智能家居, 安全监控和环境监测的理想选择. 在这些应用中, STM32 凭借其可靠性, 灵活性和丰富的功能, 成为开发者的首选微控制器.

\subsection{STM32 的分类}
STM32 有很多系列, 可以满足市场的各种需求, 上文所述内核上分有 Cortex-M0, M3, M4 和 M7 这几种, 每个内核又大概分为主流, 高性能和低功耗. 具体见表 \ref{tab:1-intro stm32 class}.

\begin{table}[H]
    \centering
    \caption{STM32 的分类} \label{tab:1-intro stm32 class}
    \begin{tabular}{cccc}\toprule
        \textbf{CPU 位数} & \textbf{内核} & \textbf{系列} & \textbf{描述}    \\\midrule
        32              & Cortex-M0   & STM32-F0    & 入门级            \\
                        &             & STM32-L0    & 低功耗            \\
                        & Cortex-M3   & STM32-F1    & 基础型, 主频 72MHz  \\
                        &             & STM32-F2    & 高性能            \\
                        &             & STM32-L1    & 低功耗            \\
                        & Cortex-M4   & STM32-F3    & 混合信号           \\
                        &             & STM32-F4    & 高性能, 主频 180MHz \\
                        &             & STM32-L4    & 低功耗            \\
                        & Cortex-M7   & STM32-F7    & 高性能            \\
        \bottomrule
    \end{tabular}
\end{table}

单纯从学习的角度出发, 可以选择 F1 和 F4: F1 代表了基础型, 基于 Cortex-M3 内核, 主频为 72MHz; F4 代表了高性能, 基于 Cortex-M4 内核, 主频 180M.

之于 F1, F4 (429 系列以上) 除了内核不同和主频的提升外, 升级的明显特色就是带了 LCD 控制器和摄像头接口, 支持 SDRAM, 这个区别在项目选型上会被优先考虑. 但是从大学教学和用户初学来说,还是首选 F1 系列. 目前在市场上资料最多, 产品占有量最多的就是 F1 系列的 STM32.

\section{关于本课程}
本课程将以野火团队\footnote{\url{https://yehuosm.tmall.com}.}出版的《STM32 HAL 库开发实战指南》为参考, 并在其上编写本实验指导书. 笔者以搭载 STM32F407ZGT6 芯片的 F4 系列霸天虎开发板为例, 实现各种软件功能, 但除了核心之外的外设差异之外, 使用其他 F4 系列的开发板乃至其他诸如 F0, F1 系列都是相通的.

另外, 本课程可能需要使用 GitHub\footnote{\url{https://github.com}.} 上的内容. 为保证学习体验, 请尽可能配置科学上网工具.

本课程需要对 C/C++ 的多文件工程开发与编译有基本的了解, 对 Linux 操作系统熟悉更佳.

\subsection{关于 HAL 固件库}
HAL 固件库由 ST 公司提供, 包含了 STM32 芯片所有寄存器的控制操作, 相当于新建了介于硬件和用户软件之间的抽象层, 可以为接口外设, 实时操作系统提供简单的处理器软件接口, 屏蔽了硬件差异, 对于软件移植有极大好处.

为了保证本实验指导书的简单性, 此处并不想详细介绍 HAL 固件库的详细情况, 只需会使用即可. 至于固件库的下载会在后续章节说明.
